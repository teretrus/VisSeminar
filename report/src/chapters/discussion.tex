\section{Discussion}
\label{sec:discussion}

\paragraph{Feature selection} in general is not the only field that addresses
the large amount of features. Feature extraction and other mathematical or
statistical methods can also reduce an existing feature-set to a new smaller
feature set. A big difference between Feature selection and these methods is,
that feature selection selects a real subset of features. Other methods might
transfer features into other spaces or introduce artificial features that cannot
be linked directly to a natural feature that is easy to understand for the user.

\paragraph{Structured Feature methods} are often listed as better-perforing in
comparison to Flat Featrue methods. Most of the listing articles are aware of
the underlying structure and accept this structure as given. In these cases the
assumption is usually correct, but there are cases where the underlying
structure of features is not completely known and has to be defined. Defining a
wrong structure can lead to worse results than Flat Feature methods. 

\paragraph{Scalability} is an issue that is not addressed by many conventional
methods, because most of them need to keep the full data-set in the memory to
produce good results (\cite{Tang:14}). If the dataset is too large, only a few
techniques (i.e. Streaming Methods) can still perform good.

\paragraph{Stability} defines wheter an algorithm tends to select the same
features out of a feature set if different samples are given. This is a very
important property that is not yet being addressed by many algorithms
(\cite{Tang:14}). In some cases (e.g. bioinformatics) it is common to test the
stability of a method with separate sample-sets and only accepting a result, if
all sample-sets produce the same feature-sub-set.

\paragraph{Linked Data} among samples is hardly considered in any feature
selection method. If relationships are considered, methods usually only look at
relationships of features (i.e. Structured Methods) and ignore the possibility
of realted samples. When it comes to larger data-sets, more interaction between
data-samples can be seen (e.g. Social Media) and should also be taken into
account (\cite{Tang:14}).


