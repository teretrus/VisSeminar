\paragraph{Fisher Score}
\label{par:methods.flat.filter.fisher_score}

% Author: Silvana hat mal was rein getan

When using Fisher score as a criterion for feature selection, each feature is
rated and selected independent of the others by calculating its individual 
fisher score. The intuition behind fisher score is to select features in a way
so that in the resulting "feature space", data samples of the same class are 
close to each other, while the distances between data points of different 
classes are as big as possible. Algorithmically, this filter process is
relatively  easy to perform: after computing the fisher score for each feature, 
the features with the highest scores are selected. (How many features are chosen
and the definition of when a score is high enough is an open choice and
dependent on the application design.)

Despite its relative simplicity, this method has a major drawback. As no 
eventual correlation between the features is considered, it is likely that the 
optimal subset is not selected. For example, if two features score relatively
low  individually, they eventually would score high in combination and cause
better  classification results. For the same reason, fisher score cannot handle
redundant  features. When selecting two features, it is possible that the same
accuracy in  classification could be achieved with only one of them.

Generalized Fisher score (GFS) is an attempt to overcome this problem. Instead
of evaluating one feature after the other individually, several features are 
considered simultaneously. GFS was tested and compared with other state-of-the
art methods, including  Fisher score, and proved to perform better in
applications like face recognition  and number recognition. For face
recognition, GFS selected features  (in this case pixels) with an asymmetric
distribution over the whole  face-area, while fisher score selected many pixels
which where clustered in  non-face areas. \cite{Gu:12}
  