\paragraph{Branch and Bound}
\label{par:methods.flat.wrapper.branch_and_bound}

% Author: Silvana

Branch-and-bound (BB) is a general design principle for solving discrete and combinatorial optimization problems. BB-algorithms work by building up a search tree with possible candidate solutions (branch phase). Further, a criterion function has to be designed, which evaluates the "quality" of possible solutions represented by nodes in the tree. Then, the branches of the tree are explored to find a global optimum for the given function. To avoid that the search is exploring the whole possible space, and thus degenerating to a brute-force-approach, it is limited by so-called bounds, which prune the tree when it becomes unlikely to find an optimum in the current branch. 

For feature selection, Narendra et al. \cite{Narendra:77} first proposed a BB-algorithm based on efficient subset selection. The root of the tree represents the full set of features, whereas leaf-nodes represent single features (or subsets of the smallest desired size). Monotonicity is required regarding the criterion function and the subsets. That means, if a current subset is below a bound, then its following child nodes are assumed to be also lower than that bound. Thus, the search in this particular branch can be aborted, and continued at another branch. Additionally, the sorting of the nodes is done with an efficient enumeration scheme to augment finding an optimal solution in a short amount of time.

The problem with BB-methods is that it is not guaranteed to perform better (faster) than exhaustive search methods. It is not granted that enough sub-trees will be pruned to speed up the search sufficiently, in worst-case, the criterion-function is evaluated for every node in the tree. Additionally, evaluating the function close to the root takes longer than evaluating it at the leaves. \cite{Somol:04} proposed a fast BB approach, which tries to keep the evaluations of the criterion function at a minimum to speed up the search. The algorithm is "learning" how the removal of features influences the functions value, and tries to "predict" changes without doing the actual computation. Only if the predicted value comes close to a bound, the criterion function is evaluated, otherwise the algorithm continues descending in the tree.

Adaptive branch and bound for feature selection \cite{Nakariyakul:07} also tries to reduce the number of evaluations of the criterion function, by applying several improvements. Already when building up the tree, the order of the nodes corresponds to the importance of the features. Then, instead of trying to descend sequentially from node to node, an adaptive, "jumping" search method is used to avoid more-or less redundant computations of the evaluation function. Additionally, the initial bound used in the search and also the initial search level are chosen in an optimal way. (Comment: only tested on 30 features?! Still check this out...)
