\paragraph{Hill Climbing}
\label{par:methods.flat.wrapper.hill_climbing}

% Author: Silvana

Hill climbing, sometimes also referred to as "steepest ascend", is probably the
simplest search technique that can  be applied. Starting with no features at
all, an evaluation function rates the  available features and adds the most
relevant one to the subset. Then the  procedure is repeated, each time adding
the most relevant feature of the  remaining set to the subset. As soon as the
influence of an added feature does  not improve the overall quality of the set
significantly, the search is stopped.  The problem with hill climbing is that it
gets stuck in local maxima easily,  and fails in finding a globally optimal
solution.  \cite{Kohavi:97}

Best-first search outperforms hill-climbing and turns out to be a more robust
technique according to Kohvani  \cite{Kohavi:97}.
