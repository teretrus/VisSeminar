\paragraph{Genetic Algorithms}
\label{par:methods.flat.wrapper.genetic}

% Author: Silvana

(SOURCE!!! basically written down by what Silvana knows from se bachelor thesis...)
Before explaining how GA can be used for for feature selection, 
a short introduction into the very basic concept of those algorithms has to be made. As the name implies, 
genetic algorithms are inspired by the way nature �works� in real life: parents carry specific genetic  information, 
and a (re-)combination of this information is passed to their kids in order to create �better� offspring from generation to generation. 
GA's imitate this behaviour by encoding the data they should work on or optimize in so-called chromosomes 
(which could be for examples vectors of numerical values). 
One is not limited to using numerical data, but this is a very common approach, 
as numerical values can be processed easily by the typical routines in a GA. 
A desired number of chromosomes is initialized at the beginning of the algorithm. 
They can be created randomly (f.ex by creating vectors with random values) or according to specific rules.
As the algorithm starts, different operations are applied to create new chromosomes: 
for example, a new chromosome can be obtained by combining the values of two already existing chromosomes, 
or by changing random values in an already existing chromosome. 

The resulting new chromosomes are then evaluated according to a fitness function, 
which basically measures how �good� a chromosome is suited for the underlying problem. 
The chromosomes then can be ranked, and the best ones are taken again to create new offspring. 
This procedure is repeated as long until a desired quality/result is achieved.

Just like branch and bound algorithms, GA are generally not suited for big feature spaces, 
as they try to explore the whole search space until a stopping criterion is met. 
This takes a lot of computation time when feature spaces are highly dimensional. 
The strength of GA is that they do not tend to get stuck at local optima 
(es for example Hill climbing algorithms do), but instead are likely to find global optima.

For feature selection, the chromosomes could correspond to subsets of the total feature set. 
The values in a chromosome would encode if a feature is selected or not. This can be achieved simply by using binary values: 
1 indicates that a feature is selected for a possible subset, and 0 indicates that it is not selected. 
Mutation and recombination operators would modify the chromosomes, 
and a fitness function would evaluate their quality for the underlying classifier 
(remember, each chromosome resembles a subset of features).

But one is not limited to binary values [XY] and [BLA] used GA to find optimal kernel setting for SVM. [quote those two papers found from  2006] 

  
TODO
Bla bla bla\ldots