\section{Methods}
\label{sec:methods}

% Author: Flo

Although it is hard to make a strict classification of feature-types,
in general, three three different types of features can be distinguished: 

\begin{itemize}
  \item Flat Features
  \item Structured Features
  \item Streaming Features
\end{itemize}

The problem with making a clear classification is that the features themselves,
being regarded simply as numerical data gained by making measurements, have no
structure per se. In fact, the same set of features can be  regarded as flat or 
structured, for example. Dependent on how the underlying problem is modeled,
a certain structure can be assumed. 

The decision, which method should be taken, is therefore dependent on the assumed 
properties of the features. Flat Feature methods take no relationship of
features into account, whereas Structured Feature methods take a given
relationship and use it to preserve structural information.

Since it is not always clear how one feature is related to the other features,
structures are sometimes hard to define and might also give worse results if assumed
wrong. On the other hand, if the structure tends to represent the correct relationships
of the given features, Structured Feature methods tend to outperform Flat
Feature methods (\cite{Tang:14}).

While Flat Feature methods and Structured Feature methods assume a given, finite
feature set, Streaming Feature methods work on an infinite or growing feature
set. Streaming Feature methods are a more general approach, and find
applications in social media.

\subsection{Methods for Flat Features}
\label{sec:methods.flat}

% Author: Silvana

Flat features are features which are assumed to be independent of each other. No intrinsic or group-like structures are induced. 
For flat features, three different type of feature selection methods can be distinguished: Filter, Wrapper and Embedded methods.
In this section, a short introduction on those three categories will be given, including a discussion of their benefits and disadvantages.

\subsubsection{Filter Methods}
\label{sec:methods.flat.filter}

% Author: Silvana

As the name suggests, filter methods try to filter out relevant features from
not-relevant ones. 
This is done by considering only the actual dataset, 
the properties of the classifier which will be used afterwards are ignored.
 
Thus, filter methods are independent of the classification algorithm: Any
classifier can be used on the filtered features, 
which allows to try out different algorithms on the feature set.
 
This can be seen as an advantage, but on the other side, 
complete independency can also have its drawbacks. 
As characteristics of the specific classifiers are not considered, 
it is not clear which classifier works best with the selected subset. 
Thus the selected subset does not guarantee optimal performance of the classifier.

Quadratic programming Feature Selction:
http://jmlr.org/papers/volume11/rodriguez-lujan10a/rodriguez-lujan10a.pdf 

\paragraph{Fisher Score}
\label{par:methods.flat.filter.fisher_score}

% Author: Silvana
  
TODO
Bla bla bla\ldots
\paragraph{Information Gain}
\label{par:methods.flat.filter.information_gain}

% Author: Silvana
  
TODO
Bla bla bla\ldots
  

\paragraph{Min. Redundancy - Max. Relevance}
\label{par:methods.flat.filter.min_redundancy_max_relevance}

% Author: Flo


\paragraph{Relief}
\label{par:methods.flat.filter.relief}

% Author: Silvana
The original Relief-algorithm was presented by Kira and Rendell \cite{Kira:92}, and can be used for binary classification problems. The algorithm is efficient (it runs in polynomial time), robust and tolerant to noisy data, but the quality of the results is dependent on the number of iterations performed.

The method operates on the whole trainingset of samples and the features each sample consists of. A vector of weights is used, where each weight represents a feature. In each iteration, one of the samples (further referred to as x) is chosen at random. Then the near-hit and near-miss instance of the remaining set in relation to x are selected: The first is the most similar instance from the same class as x, and the latter is the most similar instance of the different class. The distance between the features of x and the near-hit and x and the near-miss is calculated to determine if a feature is relevant or not.  

Intuitively, if a feature is relevant, the near-hits should be closer than the near-misses on average, while for irrelevant features, near-hits and near-misses are very similar to each other and are not very discriminative. For each feature $f_i$, the corresponding weight $w_i$ in the weight-vector is updated dependent on the euclidean distances by applying equation \ref{relief}:

\begin{equation}
\label{relief}
	w_i = w_i - (x_i - nearhit_i)^2  + (x_i - nearmiss_i)^2
\end{equation}

Relevant features thus score values larger than zero, while irrelevant features become zero or negative. After a desired number of iterations, the features whose weights have a value above a certain threshold are chosen for the classification or other processing routines.

The original algorithm was further improved. ReliefF \cite{Kononenko:97} is an extension so that the algorithm is able to handle incomplete data and multiple-class problems. 
RRELIEFF (Regressional ReliefF) \cite{Robnik-Sikonja:97} adapts the algorithm to handle also linear regression problems.

The family of Relief-algorithms and eventual adaptations has successfully been used on feature selection problems in the recent years. Moore used an adaptation called Tuned ReliefF (TuRF) for genetic analysis \cite{Moore:07}, where the worst features are removed systematically in each iteration. By removing a fixed number of features in each iteration, the accuracy of weight-estimation is increasing compared to the results of the original ReliefF algorithm. Eppstein and Haake \cite{Eppstein:08} criticize that for truly genome-wide association analysis, where a huge number of features is used (up to hundreds of thousands), all the proposed methods don't scale well and the estimated weights become basically random values. Very large scale ReliefF (VLSReliefF) tries to overcome this limitation by simply applying Relief on subsets of the features, and then combining the results to gain the weights for all features. This process can be parallelized on the GPU to speed it up \cite{Lee:15}
In the context of media classification, the ReliefF algorithm was used for automated classification of websites \cite{Jin:07}.


\subsubsection{Wrapper Methods}
\label{sec:methods.flat.wrapper}

% Author: Silvana

In contrast to filter methods, wrapper methods are dependent on the classifier used after the feature selection. 
The feature-set is selected so that it fits the biases and heuristics of the classifier as good as possible. 

This is achieved by sequentially applying two steps: 

\begin{itemize}
  \item A search routine selects a set of features 
  \item The selected subset is evaluated with the desired classifier
\end{itemize}

Those two steps are repeated until a desired quality of the subset is reached. 
The first step will produce a set of features, which then are used by the classifier. 
The feature set with the best results (e.g. highest estimation values) will
then be used for the actual classification task.


TODO: Das hier ohne Doppelpunkt ausformulieren

Search routine:
The size of the search space depends on the number of features: The more there are, the more possible subsets of them exist. 
As real-life applications normally involve a lot of features, the search can not be done randomly, but has to follow a certain strategy, too. 
Typical algorithms used are hill-climbing, best-first, or greedy-search (forward selection/backward elimination paper!), to name a few.

Evaluation:
The quality of the selected subset can be evaluated by different methods, using a simple validation set or cross-validation are two common methods.


The major drawback of this approach is the computation time needed, as the subsequent search- and evaluation steps are computationally expensive. 
Compared to filter methods, better performance and more accurate estimates are achieved for the chosen classifier.
Also, as the selected subset is dependent on the classifier, it cannot be used with any classifier, as the results will be biased.


Wrapper methods were already introduced shortly in chapter FOO. 
As stated, they consist of two different steps, a search- and a evaluation-step, 
which are applied sequentially in each iteration until a stopping criterion (usually sufficient quality in classification) is met.

This chapter will go more in-depth about different search-techniques, before discussing validation-strategies shortly. 


TODO
Bla bla bla\ldots

\paragraph{Hill Climbing}
\label{par:methods.flat.wrapper.hill_climbing}

% Author: Silvana

Hill climbing, sometimes also referred to as "greedy search" or "steepest ascend", is probably the simplest search technique that can be applied. Starting with no features at all, an evaluation function  rates the available features and adds the most relevant one to the subset. Then the procedure is repeated, each time adding the most relevant feature of the remaining set to the subset. As soon as the influence of an added feature does not improve the overall quality of the set significantly, the search is stopped. The problem with hill climbing is that it gets stuck in local maxima easily, and fails in finding a globally optimal solution. \cite{Kohavi:97}
\paragraph{Best Fit}
\label{par:methods.flat.wrapper.best_fit}

% Author: Silvana
  
TODO
Bla bla bla\ldots
\paragraph{Branch and Bound}
\label{par:methods.flat.wrapper.branch_and_bound}

% Author: Silvana

...usw
  
  
TODO
Bla bla bla\ldots
\paragraph{Genetic Algorithms}
\label{par:methods.flat.wrapper.genetic}

% Author: Silvana

(SOURCE!!! basically written down by what Silvana knows from se bachelor thesis...)
Before explaining how GA can be used for for feature selection, 
a short introduction into the very basic concept of those algorithms has to be made. As the name implies, 
genetic algorithms are inspired by the way nature �works� in real life: parents carry specific genetic  information, 
and a (re-)combination of this information is passed to their kids in order to create �better� offspring from generation to generation. 
GA's imitate this behaviour by encoding the data they should work on or optimize in so-called chromosomes 
(which could be for examples vectors of numerical values). 
One is not limited to using numerical data, but this is a very common approach, 
as numerical values can be processed easily by the typical routines in a GA. 
A desired number of chromosomes is initialized at the beginning of the algorithm. 
They can be created randomly (f.ex by creating vectors with random values) or according to specific rules.
As the algorithm starts, different operations are applied to create new chromosomes: 
for example, a new chromosome can be obtained by combining the values of two already existing chromosomes, 
or by changing random values in an already existing chromosome. 

The resulting new chromosomes are then evaluated according to a fitness function, 
which basically measures how �good� a chromosome is suited for the underlying problem. 
The chromosomes then can be ranked, and the best ones are taken again to create new offspring. 
This procedure is repeated as long until a desired quality/result is achieved.

Just like branch and bound algorithms, GA are generally not suited for big feature spaces, 
as they try to explore the whole search space until a stopping criterion is met. 
This takes a lot of computation time when feature spaces are highly dimensional. 
The strength of GA is that they do not tend to get stuck at local optima 
(es for example Hill climbing algorithms do), but instead are likely to find global optima.

For feature selection, the chromosomes could correspond to subsets of the total feature set. 
The values in a chromosome would encode if a feature is selected or not. This can be achieved simply by using binary values: 
1 indicates that a feature is selected for a possible subset, and 0 indicates that it is not selected. 
Mutation and recombination operators would modify the chromosomes, 
and a fitness function would evaluate their quality for the underlying classifier 
(remember, each chromosome resembles a subset of features).

But one is not limited to binary values [XY] and [BLA] used GA to find optimal kernel setting for SVM. [quote those two papers found from  2006] 

  
TODO
Bla bla bla\ldots
\paragraph{Forward Selection/backward elimination}
\label{par:methods.flat.wrapper.forward_selection}

% Author: Silvana
When using a greedy algorithm in the search-step of a wrapper algorithm, two strategies can be followed when selecting the features:
sequential forward selection (SFS) and sequential backward elimination (SBE).
 
SFS starts with an empty feature set, and adds one feature in each iteration, aiming to create a better subset in each iteration.
(The evaluation step is oftern done by cross validation, as mentioned in the previous chapter.) 
SBE works the other way round: in the first iteration, the full set of features is used and evaluated. Features are then deleted sequentially, 
until a smaller subset of sufficient quality is gained. 

In practice, forward selection is used more often, as it is computationally more efficient to train the classifier often with smaller subsets 
in the beginning, than performing many trainings with big feature sets.

Both methods have a major drawback: Either features cannot be eliminated once they have been selected (SFS), or they cannot be selected again 
if they have been discarded once (SBE). For example, if ten features are selected and perform better compared to all the previous subsets,
it is still possible that one feature could be replaced by one which performs even better in combination with the other nine. 
In other words: the assumption, that the best $x$ selected features must contain a subset of the best $x-1$ selected features does not hold in practice.
\cite{Nakariyakul:08}

\cite{Pudil:94} proposes Sequential forward floating selection as improvement (and, respectively, Selective backward floating elimination). 
A backtracking step is implemented after each sequential addition or deletion, and tries to find eventually better subsets. Both methods 
show admissible computation-time for small- or medium scaled problems, and perform better than other sequential methods on a variety of
problems. However, it should be empathized that they do not always perform better than other methods, but at least "good enough" on the 
majority of problems. For very big datasets, they are outperformed by genetic algorithms.\cite{Kudo:00}   

\cite{Mao:04} proposes \textit{orthogonal} forward selection and backward elimination to overcome problems that occur with SFS and SBE. 
Instead of simply selecting features in a sequential way, they are first mapped to an orthogonal space. 
This mapping decorrelates the features, so they can be evaluated and selected individually. 
After the selection, the features are linked back to the same number of variables in the original measurement space.
Using a mapping to orthogonal space proved to be very effective for features with high correlation. 
If the correlation between candidate features is only trivial, orthogonal transforms don't improve the results compared to existing sequential methods.
\paragraph{Greedy}
\label{par:methods.flat.wrapper.greedy}

% Author: Silvana
  
TODO
Bla bla bla\ldots
\subsubsection{Embedded Methods}
\label{sec:methods.flat.embedded}

 % Author: Flo
 
Embeddd Methods are an approach to combine the computational efficiency of
Filter Methods with the quality of Wrapper Methods.
By addressing the characteristics of the classifier, Embedded Methods select
features that are more suiting for classification than Filter Methods. At the
same time, they are not as computational expensive as Wrapper Methods.
  
Embedded Methods can roughly been characterized into three categories:

\begin{itemize}
  \item Pruning Methods
  \item Built-in Methods
  \item Regression Methods
\end{itemize}

 
\paragraph{Pruning}
\label{par:methods.flat.embedded.pruning}

% Author: Flo
  
TODO
Bla bla bla\ldots
\paragraph{Recursive Feature Elimination using Support Vector Machines}
\label{par:methods.flat.embedded.svm}

% Author: Flo

is a Pruning Method, that uses SVM(Support Vector Machines) to select the most
contributing features.

Since SVM-classifiers are basically a hyperplane that separates data in a
high-dimensional space into positive and negative matches, the classification
can be reduced to a simple inside-outside test with the plane, wich is basically
a dot-product with the corresponding hesse-normal-form of the plane.

This method now eliminates all components of the plane, in which the
hesse-normal-form of the plane is close to $0$, and therefore removing
less- and non-contributing features \cite{Brank:02}.

As SVMs mostly work in a space which has a higher dimensionality than the
original feature-space, the remaining features of the high dimensional space
need to be mapped to features of the low dimensional space to effectively
eliminate features.

\paragraph{Built-in Methods}
\label{par:methods.flat.embedded.builtin}

% Author: Flo
  
use an approach that is fundamentally different to other
methods.

\paragraph{C 4.5}
\label{par:methods.flat.embedded.c4_5}

% Author: Flo
  
TODO
Bla bla bla\ldots
\paragraph{ID3}
\label{par:methods.flat.embedded.id3}

% Author: Flo
  
TODO
Bla bla bla\ldots

\paragraph{Regression Methods}
\label{par:methods.flat.embedded.regression}

% Author: Flo

try to analytically force coefficients to be small while
still fitting the model. After this computation, coefficients close to $0$ can
be eliminated.

In general, to classify a feature-vector, Regression Methods minimize an
expression where an error term adds up with a penalty (\cite{Tang:14}). Since
the error term can usually be controlled by the user (typical error terms, such as squared
distance can be used), Regression Methods differ in the penalty-term.

Some examples of Regression Methods are:

\begin{itemize}
  \item Lasso Regularization (\cite{Tibshirani:96})
  \item Adaptive lasso (\cite{Zou:06})
  \item Bridge Regularization (\cite{Knight:00},\cite{Huang:08})
  \item Elastic Net Regularization (\cite{Zou:05})
\end{itemize}

The Lasso Regularization is the basic algorithm, that is used by other
Regression Methods.
\subsection{Methods for Structured Features}
\label{sec:methods.structured}

% Author: Flo

TODO
Bla Bla\ldots es gibt die und die methods\ldots

\begin{itemize}
  \item Graph methods
  \item Tree methods
  \item Group methods
\end{itemize}


\subsection{Graph Methods}
\label{sec:methods.structured.graph}

% Author: Flo
  
\subsubsection{Tree Methods}
\label{sec:methods.structured.tree}

% Author: Flo

With Tree Methods, features are assumed to have a certain structure, where
features can be grouped into groups, and these groups can again be grouped into
groups, until there is only one group left.

This index-tree-structure can be visualized as tree, with all features being
leafes (see Figure \ref{fig:methods.structured.tree.lasso}).

\begin{figure}[!ht]
  \centering 
  \includegraphics[width=0.5\textwidth]{chapters/methods/structured/tree_lasso}
  \caption{Reprinted from \cite{Tang:14}. Illustration of an index tree.
  E.g. Features $f_1$ and $f_2$ can be grouped into $G_1^2$ (\cite{Tang:14}).}
  \label{fig:methods.structured.tree.lasso}
\end{figure}

Using index-trees, methods like the tree structured group Lasso (\cite{Kim:10})
can use this structure to eliminate tree-nodes on a higher level of the hierachy and
therefore eliminate many features at once.
\subsection{Group Methods}
\label{sec:methods.structured.group}

% Author: Flo
  
TODO
Bla bla bla\ldots
\subsection{Methods for Streaming Features}
\label{sec:methods.streaming}

% Author: Silvana

TODO
Bla Bla\ldots 


\paragraph{Gender Prediction on Twitter}
\label{sec:methods.streaming.gender}

% Author: Silvana

TODO
Bla bla\ldots

	
